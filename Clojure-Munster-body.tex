\mode<presentation> {
% \usetheme{JuanLesPins}
  \usetheme{Madrid}
%  \usetheme{Rochester}
  \setbeamercovered{transparent}
%\usefonttheme[onlylarge]{structuresmallcapsserif}
\usefonttheme[onlysmall]{structurebold}
}

%\usepackage[a4paper,hmargin={2.5cm,2cm},vmargin={2cm,2cm},nohead,twoside]{geometry}
\usepackage{mathtext}
\usepackage{ucs}
\usepackage[utf8x]{inputenc}
\usepackage[T2A]{fontenc}
\usepackage[english]{babel}
\hypersetup{unicode,colorlinks=true,bookmarks}

\usepackage{listings}
\input{clojure-def}

%http://old.nabble.com/latex-beamer-users-f4031.html

\title{Clojure}
\subtitle{A small introduction}
\author{Alex Ott}
\institute[\lstinline!alexott@gmail.com!]{\url{http://alexott.net}}
\date{Münster JUG, July 2010}

\begin{document}

\selectlanguage{english}

\frame{\titlepage
}

\begin{frame}[t]
  \frametitle{About me...}
  \begin{itemize}
  \item Professional software developer \& architect
  \item Working with different programming languages -- C/C++, Lisps, Haskell, Erlang, ...
  \item About 20 years of Lisp experience, including commercial projects
  \item Development on Unix/Linux platforms
  \item Author of articles on programming (including Clojure), Emacs, etc.
  \item Participate in many open source projects, including Incanter, labrepl, etc.
  \item Maintainer of ``Planet Clojure''
  \end{itemize}
  \note{First, I want to say several words about myself. My name is Alex
    Ott. I'm professional software developer \& architect, although physicist by
    education. Last 10 years I worked in content security branch, and now
    working in McAfee, as Sr. Developer.

    I'm using many languages in my work, some for prototypes, some for
    production. 

    I mostly working on Unix-like platforms -- Linux, Mac OS X, etc.

    I've used Java starting from version 1.0, and periodically until 2005th

    I participate in many open source projects, including clojure-related. I'm
    also author of many articles.  }
\end{frame}


\begin{frame}[t]
\frametitle{Outline}
\tableofcontents
\note{ So, what we'll talk about:

  In first part I'll speak about why new language was created, its features, and
  differences from other Lisp dialects.

  Second part will provide introduction into syntax of language

  In third part I'll describe interoperability questions -- how to call Java
  methods, how to create Java objects, etc.  I should to mention, that we could
  use not only JVM, but also .Net, in Clojure.Net version (beta now)

  In forth part we'll look onto concurrent programming, and which Clojure's
  features are used for it

  And in last parts, we'll look onto real-life questions -- infrastructure \&
  tools, sources of informations, look onto some examples, etc.  }
\end{frame}

\section{What is Clojure?}

\begin{frame}[t]
  \frametitle{What is  Clojure?}
  \begin{columns}
    \begin{column}{0.7\textwidth}
  \begin{itemize}
  \item Lisp-like language, created by Rich Hickey and announced in 2007th
  \item Designed to work on existing platforms
  \item Functional programming language, with immutable data
  \item Special features for concurrent programming
  \item Open sourced with liberal license
  \item Already used in many projects, including commercial
  \end{itemize}
    \end{column}
    \begin{column}{0.3\textwidth}
      \begin{center}
        \pgfimage[width=2.5cm]{images/clojure}
      \end{center}
    \end{column}
  \end{columns}
  \note{JVM (and .Net) was selected because these platforms have very good
    garbage collectors, support JIT-compilation, etc. Plus, for these platforms
    we have access to big number of existing libraries.  And usage of given
    platform is often customer's requirement!  

    Functional languages -- easy to reason about programs, easy to test.  First class
    functions.  Functions free from side-effects and always produce same result.
    Side-effect-free function is possible to execute in parallel automatically.

}
\end{frame}

%TODO: add slide 'Why functional programming'?

\begin{frame}[t]
  \frametitle{Why new language?}
  \begin{columns}
    \begin{column}{0.75\textwidth}
  \begin{itemize}
  \item Lisp, but without compatibility with previous versions
  \item Immutable data and more support for pure functional programming comparing with
    other Lisps
  \item Support for concurrent programing inside language
  \item Better integration with target platforms: JVM~\&~.Net
  \end{itemize}
    \end{column}
    \begin{column}{0.25\textwidth}
      \begin{center}
        \pgfimage[width=2.5cm]{images/question4}
      \end{center}
    \end{column}
  \end{columns}

\note{
Existing Lisps maintains compatibility with standards (and previous
versions), so we couldn't introduce breaking changes for them

Common Lisp and Scheme are imperative languages with some functional parts

To make proper concurrency support, it's better to introduce it in
language, not library, level.

Transparent integration with target platforms is very important. Existing
JVM support in other languages is often mimic Java....

}

\end{frame}

% ???
\begin{frame}[t]
  \frametitle{Main features}
  \begin{itemize}
  \item Dynamically typed language
  \item Very simple syntax, like in other Lisps (code as data)
  \item Support for interactive development
  \item Designed in terms of abstractions
  \item Metaprogramming
  \item Multimethods and protocols (in version 1.2)
  \item Compiled into byte code of target platforms
  \item Access to big number of available libraries
  \end{itemize}
\note{

  Interactive development -- REPL, define functions on the fly, load and
  compile code in runtime. Data introspection.

  Metaprogramming allows to build DSL that will compiled into effective
  code.

  Older Lisps were more close to machine.  Clojure provides abstractions
  for many things -- collections, threads, etc.  This allows to provide
  uniform interfaces with less operations.

}
\end{frame}


\begin{frame}[t]
  \frametitle{Differences from other Lisps}
  \begin{itemize}
  \item Changes in syntax, less parentheses
  \item Changes in naming
  \item More first-class data structures -- maps, sets, vectors
  \item Immutable, by default, data
  \item Ability to link meta-information with variables and functions
  \item ``Lazy'' collections
  \item There is no reader macros
  \item Case-sensitive names
  \item Lack of tail call optimization (JVM's limitation) -- explicit loops \texttt{loop/recur}
  \item Exceptions instead of signals and restarts
  \end{itemize}
  \note{

    Mention, that Clojure is Lisp-1, without separate namespaces for variable,
    functions, etc.}
\end{frame}


\section{Language basics}

\begin{frame}[t]
  \frametitle{Base data types}
  \begin{itemize}
  \item Integer numbers of any size -- \texttt{14235344564564564564}
  \item Rational numbers -- \texttt{26/7}
  \item Real numbers -- \texttt{1.2345} and BigDecimals -- \texttt{1.2345M}
  \item Strings (\texttt{String} class from Java) -- \texttt{"hello world"}
  \item Characters (\texttt{Character} from Java) -- \texttt{$\backslash{}$a}, \texttt{$\backslash{}$b}, \ldots{}
  \item Regular expressions -- \texttt{\#"[abc]*"}
  \item Boolean -- \texttt{true} \& \texttt{false}
  \item \texttt{nil} -- the same as \texttt{null} in Java, not an empty list as in Lisp
  \item Symbols -- \texttt{test}, \texttt{var1}, \ldots{}
  \item Keywords -- \texttt{:test}, \texttt{:hello}, \ldots{}
  \end{itemize}
\note{discuss each type...}
\end{frame}

\begin{frame}[t,fragile]
  \frametitle{Data structures}
  \begin{itemize}
  \item Separate syntax for different collections:
    \begin{itemize}
    \item Lists -- \texttt{(1 2 3 "abc")}
    \item Vectors -- \texttt{[1 2 3 "abc"]}
    \item Maps -- \lstinline!{:key1 1234 :key2 "value"}!
    \item Sets -- \lstinline!#{:val1 "text" 1 2 10}!
    \end{itemize}
  \item Sequence -- abstraction to work with all collections (including classes from
    Java/.Net)
  \item Common set of functions to work with sequences
  \item Lazy operations on sequences
  \item Vectors, maps, and sets are functions -- to simplify access to data in them
  \item Persistent collections
  \item Transient collections -- performance optimization
  \item \texttt{defstruct} -- optimization of map to work with complex data structures
  \end{itemize}
  \note{

    Because all data is immutable, all functions should return new object.  But in
    Clojure, like in other functional languages, data isn't copied completely, but parts
    of them are shared when modified copy is created.  Mention ``Purely functional data
    structures book'' when speaking about persistent data structures.

    \texttt{defstruct} is used to optimize performance, when you use many maps
    with similar structures.  But structure isn't fixed!

}
\end{frame}

\begin{frame}[t]
  \frametitle{Syntax}
  \begin{itemize}
  \item Very simple, homoiconic syntax  -- program is usual data structure
  \item Special procedure (reader) process text and produce data structures
  \item All objects represent themselves except symbols and lists
  \item Symbols link value with ``variable''
  \item Lists represent forms, that could be:
    \begin{itemize}
    \item Special form -- \texttt{def}, \texttt{let}, \texttt{if}, \texttt{loop}, \ldots{}
    \item Macros
    \item Function call or expression, that could be used as function\\
      (maps, keywords, returned functional values, \ldots{})
    \end{itemize}
  \item To get list as-is you need to ``quote'' it (with \texttt{quote} or \lstinline|'|)
  \end{itemize}
  \note{Special form is a form, where normal arguments evaluation order isn't
    working.}
\end{frame}

\begin{frame}[fragile,fragile]
  \frametitle{Code example}
\begin{lstlisting}
(defn fibo
  ([] (concat [1 1] (fibo 1 1)))
  ([a b]
     (let [n (+ a b)]
       (lazy-seq (cons n (fibo b n))))))
(take 100000000 (fibo))

(defn vrange2 [n]
  (loop [i 0 
         v (transient [])]
    (if (< i n)
      (recur (inc i) (conj! v i))
      (persistent! v))))
\end{lstlisting}
\note{go through code}
\end{frame}

\begin{frame}
  \frametitle{Life cycle of Clojure program}
  \begin{itemize}
  \item Code is read from file, or REPL, or from other code
  \item Reader transforms program into data structures
  \item Macros are expanded
  \item Resulting code is evaluated/compiled
  \item Produced bytecode is executed by JVM
  \end{itemize}
  \begin{center}
  \pgfimage[width=8cm]{images/Clojure-Lifecycle}
  \end{center}
\note{To proper understand how Clojure code is processed, let look on
  following picture:

You can see, that code could be received from different sources, then it
processed by reader procedure, that converts it into data structures, that
will compiled into bytecode (after macro expansion)

This scheme differs from traditional Java approach -- Edit/Compile/Pack/Run

}
\end{frame}

\begin{frame}[t,fragile]
  \frametitle{Functions}
  \begin{itemize}
  \item Functions are first-class objects
  \item Function can have variable number of arguments % см. предыдущий слайд, функция fib
  \item Function's definition: \lstinline!defn! -- top-level functions, \lstinline!fn! --
    anonymous functions
  \item Simplified syntax for anonymous functions -- \lstinline!#(code)!. For example:\\
    \lstinline!(map #(.toUpperCase %) ["test" "hello"])!\\
    \lstinline!(map #(vector %1 %2) [1 2 3] [4 5 6])!
  \item You can specify tests \& pre/post-conditions.  For example,
\begin{lstlisting}
(defn constrained-sqr [x]
  {:pre  [(pos? x)] :post [(> % 16), (< % 225)]}
  (* x x))
\end{lstlisting}
  \item Each function is compiled into separate Java class
  \item Each function implements Runnable interface
  \end{itemize}
\note{

  First-class objects means, that you can accept functions as arguments, and return them
  as result of other functions

  Some objects are also working as functions -- keywords, maps/vectors/sets...

  Implementation of Runnable interface means, that you can run any function as separate
  thread.

}
\end{frame}

\begin{frame}[t,fragile]
  \frametitle{Metaprogramming \& macros}
  \begin{itemize}
  \item Macros receives source code and returns new code, that will compiled
  \item Big part of language is written using macros
  \item Variable number of arguments (same as functions)
  \item Code could be generated with list functions, but it's much handy to use
    quasi-quote -- \lstinline!`! and substitution operators -- \lstinline!~! and
    \lstinline!~@!
  \item The \lstinline!#! suffix in names is used to generate unique names
  \item \texttt{macroexpand-1} \& \texttt{macroexpand} are used to debug macros
  \end{itemize}
  \note{\lstinline!`! also "normalizes" symbols -- make them fully-qualified

    Show difference between 'tilda' \& 'tilda at'example 

}
\end{frame}

\begin{frame}[fragile,t]
  \frametitle{Macros examples (1)}
The standard macros \texttt{when}:
\begin{lstlisting}
(defmacro when
  [test & body]
  (list 'if test (cons 'do body)))  
\end{lstlisting}
when used as:
\begin{lstlisting}
(when (pos? a) 
   (println "positive") (/ b a))
\end{lstlisting}
will expanded into:
\begin{lstlisting}
(if (pos? a) 
  (do 
    (println "positive") 
    (/ b a)))
\end{lstlisting}
\note{go through example}
\end{frame}

\begin{frame}[fragile,t]
  \frametitle{Macros examples (2)}
The standard macros \texttt{and}
\begin{lstlisting}
(defmacro and
  ([] true)
  ([x] x)
  ([x & next]
   `(let [and# ~x]
      (if and# (and ~@next) and#))))  
\end{lstlisting}
will expanded into different code, depending on number of arguments:
\begin{lstlisting}
user> (macroexpand '(and ))        ==> true
user> (macroexpand '(and (= 1 2))) ==> (= 1 2)
user> (macroexpand '(and (= 1 2) (= 3 3))) ==>
(let* [and__4457__auto__ (= 1 2)] 
  (if and__4457__auto__ 
     (clojure.core/and (= 3 3)) 
     and__4457__auto__))
\end{lstlisting}
\note{go through example}
\end{frame}

\begin{frame}[fragile,t]
  \frametitle{Polymorphism \& multimethods}
  \begin{itemize}
  \item Extensibility
  \item Multimethods aren't bound exclusively to types/classes
  \item Dispatching is performed using results of user-specified dispatch function
  \item Dispatching on several parameters
  \item Differs from CLOS -- absence of \texttt{:before}, \texttt{:after}, \ldots{}
  \item \texttt{defmulti} -- is analog of \texttt{defgeneric} in Common Lisp
  \item Defined as \lstinline!(defmulti func-name dispatch-fn)! + set of implementations
    (via \texttt{defmethod})
  \item You can also define hierarchical relationships with \texttt{derive}, and use them
    in dispatch (via \texttt{isa?})
  \end{itemize}
  \note{

    Although, Clojure doesn't support object hierarchies, you anyway can build
    hierarchical relationships with (derive child parent). Child and parent can be either
    symbols or keywords, and must be namespace-qualified.

    \texttt{isa?} also tests for class relationships - the same with
    \texttt{parents}/\texttt{ancestors}

    Say about new library, that also allows to emulate \texttt{:before}, ... in
    CLOS 

}
\end{frame}

\begin{frame}[fragile,t]
  \frametitle{Multimethods examples (1)}
Simple dispatching using data type/class:
\begin{lstlisting}
(defmulti m-example class)
(defmethod m-example String [this]
  (println "This is string '" this "'"))
(defmethod m-example java.util.Collection [this]
  (print "This is collection!"))
\end{lstlisting}
and we'll get:
\begin{lstlisting}
(m-example "Hello") ==> "This is string 'Hello'"
(m-example [1 2 3]) ==> "This is collection!"
(m-example '(1 2 3)) ==> "This is collection!"
\end{lstlisting}
\note{go through example}
\end{frame}

\begin{frame}[fragile]
  \frametitle{Multimethods examples (2)}
\begin{lstlisting}
(defmulti encounter 
     (fn [x y] [(:Species x) (:Species y)]))
(defmethod encounter [:Bunny :Lion] [b l] :run-away)
(defmethod encounter [:Lion :Bunny] [l b] :eat)
(defmethod encounter [:Lion :Lion] [l1 l2] :fight)
(defmethod encounter [:Bunny :Bunny] [b1 b2] :mate)
(def b1 {:Species :Bunny})
(def b2 {:Species :Bunny})
(def l1 {:Species :Lion})
(def l2 {:Species :Lion})

(encounter b1 b2) ==> :mate
(encounter b1 l1) ==> :run-away
(encounter l1 b1) ==> :eat
(encounter l1 l2) ==> :fight
\end{lstlisting}
  \note{More complex example of dispatching on several arguments}
\end{frame}


\begin{frame}[t]
  \frametitle{Protocols \& Datatypes}
  \begin{itemize}
  \item Introduced in version 1.2 (will released shortly)
  \item Much faster than multimethods
  \item Allows to write ``Clojure in Clojure''
  \item Dispatching is done only on data types
  \item Similar to type classes in Haskell
  \item Java interface is created for each protocol
  \item \texttt{defrecord} \& \texttt{deftype} define new data types
  \item \texttt{extend-protocol} \& \texttt{extend-type} bind protocol(s) with data types
    (not only with defined in Clojure!)
  \item \texttt{reify} is used to implement protocols and interfaces for ``once-used
    types''
  \end{itemize}
\note{...}
\end{frame}

\begin{frame}[fragile]
  \frametitle{Protocol's examples}
\begin{lstlisting}
(defprotocol Hello "Test of protocol"
  (hello [this] "hello function"))

(defrecord B [name] Hello
  (hello [this] (str "Hello " (:name this) "!")))
(hello (B. "User")) ==> "Hello User!"

(extend-protocol Hello String
              (hello [this] (str "Hello " this "!")))
(hello "world") ==> "Hello world!"

(extend-protocol Hello java.lang.Object
              (hello [this] (str "Hello '" this 
                     "'! (" (type this) ")")))
(hello 1) ==> "Hello '1'! (class java.lang.Integer)"
\end{lstlisting}
\note{discuss example}
\end{frame}

\begin{frame}[fragile,t]
  \frametitle{Miscellaneous}
  \begin{itemize}
  \item Metadata -- \lstinline!#^{}! in 1.0 \& 1.1, or simply \lstinline!^{}! in~1.2
    \begin{itemize}
    \item Optional type specification (type hints) -- \lstinline!#^Type! or \lstinline!^Type!
    \item You can specify tests directly in function's declaration
    \item Scope management
    \item Access and change of metadata from code
    \item Don't change value equality
    \end{itemize}
  \item Namespaces:
    \begin{itemize}
    \item are first-class objects
    \item are used to organize code into libraries
    \end{itemize}
  \item Data destructuring (function arguments, etc.)
\begin{lstlisting}
(let [[a b & c :as e] [1 2 3 4]] [a b c e]) 
   ==> [1 2 (3 4) [1 2 3 4]]
(let [{:keys [a b c] :as m :or {b 3}} {:a 5 :c 6}]
  [a b c m]) ==> [5 3 6 {:a 5 :c 6}]
\end{lstlisting}
  \end{itemize}
\note{not so much time}
\end{frame}

%TODO: написать во что компилируется кложурный код....

\section{Interoperability with Java}

\begin{frame}[t]
  \frametitle{Interoperability with Java}
  \begin{itemize}
  \item Two-way interoperability with target platform:
    \begin{itemize}
    \item Creation of instances  -- \texttt{new} or \texttt{Class.}
    \item Call of Java methods -- \texttt{.}, \texttt{..}, \texttt{doto}
    \item Class \& interface generation -- \texttt{gen-class}, \texttt{gen-interface}, or
      \texttt{proxy} (anonymous class)
    \end{itemize}
  \item You can execute Clojure code in Java programs
  \item Separate functions to work with Java arrays:
    \begin{itemize}
    \item \texttt{make-array} -- array creation
    \item \texttt{aget}/\texttt{aset} -- access to array's elements
    \item \texttt{amap}/\texttt{areduce} -- iteration over array's elements
    \end{itemize}
  \item \texttt{memfn} allows to use Java methods as arguments of \texttt{map},
    \texttt{filter}, etc.
  \item The \texttt{set!} special form to set values inside class
  \item Generation and catch of exception -- \texttt{throw} \& \texttt{catch}
  \end{itemize}
  \note{

    The main forms are \texttt{.} and \texttt{new}, all other things just a
    syntactic sugar...  

}
\end{frame}


\begin{frame}[fragile]
  \frametitle{Interoperability examples}
  \begin{itemize}
  \item Instance creation:
\begin{lstlisting}
(new java.util.Date) <==> (java.util.Date.)
\end{lstlisting}
  \item Call of Java methods and access to data:
\begin{lstlisting}
(.substring "Hello World" 0 5) ==> "Hello"
(. "Hello World" substring 0 5) ==> "Hello"
Math/PI  ==> 3.141592653589793
(Integer/parseInt "42") ==> 42
\end{lstlisting}
  \item \texttt{..} is used to chained calls:
\begin{lstlisting}
(.. System getProperties (get "os.name")) ==> "Mac OS X"
System.getProperties().get("os.name")
\end{lstlisting}
  \item \texttt{doto} -- call of several methods for one object:
\begin{lstlisting}
(doto (java.util.HashMap.)
 (.put "a" 1) (.put "b" 2))
\end{lstlisting}
  \end{itemize}
  \note{

    In chained call example: ``Who told, that Lisp has more parentheses as
    Java?'' ;-)

}
\end{frame}

% \begin{frame}
%   \frametitle{Генерация классов}
%TODO: gen-class, gen-interface, proxy - когда что используется
%TODO: add example of generated class? or just use proxy example?
% \end{frame}

\section{Concurrent programming}

\begin{frame}[t,fragile]
  \frametitle{Concurrent programming}
  \begin{itemize}
  \item Special language features, that provides data mutation:
    \begin{itemize}
    \item Refs -- synchronous, coordinated change
    \item Agents -- asynchronous, not coordinated
    \item Atoms -- synchronous, not coordinated
    \item Vars -- local for current thread
    \item \lstinline|@| (\texttt{deref}) is used to get access to data
    \end{itemize}
  \item Parallel code execution
    \begin{itemize}
    \item \texttt{future}
    \item \texttt{pmap}, \texttt{pvalues} \& \texttt{pcalls}
    \item Native threads
    \end{itemize}
  \item Synchronization with \texttt{promise}
  \end{itemize}
\note{just high level overview}
\end{frame}

\begin{frame}[t]
  \frametitle{State and Identity}
  \begin{itemize}
  \item Clojure separates concepts of state and identity
  \item State (value) isn't changed!
  \item More information at \url{http://clojure.org/state}
  \end{itemize}

  \begin{center}
  \pgfimage[width=7cm]{images/Clojure-State-Identity}
  \end{center}

  \note{Clojure separates concepts of state of the object, and it's identity.
    State is an actual value of object, while Identity is a ``name'' (or
    reference) to actual data.  Separation of these concepts simplify
    understanding of how Clojure works in part of mutable data.

    As you can see from diagram, if we have a mutable object (ref/agent/atom),
    then different parts of code, could have access to different values
    (states), although all of them are working with the same variable (identity)
  }
\end{frame}

\begin{frame}[fragile,t]
  \frametitle{Refs \& STM}
  \begin{itemize}
  \item Coordinated change of data from several threads
  \item Software Transaction Memory provides atomicity, consistency, and isolation
  \item Changes could be made only inside transaction
  \item You can add function for data validation
  \item Ability to add watcher function
  \end{itemize}
\begin{lstlisting}
(def counters (ref {}))
(defn get-counter [key]
  (@counters key 0))
(defn increment-counter [key]
  (dosync
   (alter counters assoc key 
          (inc (@counters key 0)))))
\end{lstlisting}
\note{more discussion here}
\end{frame}

\begin{frame}[fragile,t]
  \frametitle{Agents}
  \begin{itemize}
  \item Asynchronous change of data -- ``fire and forget''
  \item Thread pools for data update functions: \texttt{send} -- for
    ``fast'' functions, and \texttt{send-off} -- for ``heavyweight'' functions
    (separate thread pool)
  \item \texttt{send} \& \texttt{send-off} get function that will applied to agent's
    current state
  \item Validators and watchers
  \item You can wait until finish of all updates
  \end{itemize}
\begin{lstlisting}
(def acounters (agent {}))
(defn increment-acounter [key]
  (send acounters assoc key 
        (inc (@counters key 0))))
(defn get-acounter [key]
  (@acounters key 0))
\end{lstlisting}
\note{say about thread pools, etc.}
\end{frame}

\begin{frame}[t,fragile]
  \frametitle{Vars \& Atoms}
  \begin{itemize}
  \item Atoms
    \begin{itemize}
    \item Synchronous data change without coordination
    \item Data are changed by applying function to current state of atom
    \item Function could be called several times
    \item Function shouldn't have side effects!
    \end{itemize}
  \item Vars
    \begin{itemize}
    \item Provide data change only in current thread
    \item \texttt{binding} links new values with symbols
    \item Change affects all functions, called from current thread
    \item Be careful with lazy sequences!
    \end{itemize}
\begin{lstlisting}
(def *var* 5)
(defn foo [] *var*)
(foo) ==> 5
(binding [*var* 10] (foo)) ==> 10
\end{lstlisting}
  \end{itemize}
\note{not so much time here}
\end{frame}

\begin{frame}[t,fragile]
  \frametitle{Parallel code execution}
  \begin{itemize}
  \item \texttt{future}
    \begin{itemize}
    \item executes given code in separate thread
    \item \lstinline|@| is used to get results of code execution
    \item \lstinline|@| blocks current thread, if results aren't available yet
    \item results of code execution are cached
    \end{itemize}
  \item \texttt{promise}
    \begin{itemize}
    \item is used for synchronization between parts of programs
    \item \texttt{deliver} sets value of \texttt{promise}
    \item \lstinline|@| reads value, or blocks execution if value isn't set yet
    \end{itemize}
  \item \texttt{pmap}, \texttt{pvalues} \& \texttt{pcalls} are used for
    ``heavyweight'' operations, that could be performed in parallel
  \end{itemize}
  \note{

    Code, executed in future is evaluated only once, after finishing results are
    cached and accessible for any number of times}
\end{frame}

\section{Clojure in real life}

\begin{frame}[t]
  \frametitle{Clojure in real life}
  \begin{itemize}
  \item Existing infrastructure:
    \begin{itemize}
    \item IDEs
    \item Build tools
    \item Debugging and related tools
    \item Libraries
    \item Code repositories
    \end{itemize}
  \item Clojure/core -- commercial support \& consulting
  \item Clojure is used in commercial projects since 2008
  \item Sources of information
  \end{itemize}
\note{not so much time here}
\end{frame}

\begin{frame}[t]
  \frametitle{Infrastructure: IDEs and build tools}
  \begin{itemize}
  \item Supported by most of popular IDEs:
  \begin{itemize}
  \item Emacs + SLIME (most popular now)
  \item VimClojure
  \item NetBeans
  \item Eclipse
  \item IntelliJ IDEA
  \end{itemize}
  \item Build tools:
    \begin{itemize}
    \item Clojure support in Maven, Ant and Gradle
    \item Leiningen -- written in Clojure, extensible and very simple
    \end{itemize}
  \end{itemize}
\note{go through each option...

say, that VimClojure author is here
}
\end{frame}

\begin{frame}[t]
  \frametitle{Infrastructure: libraries \& repositories}
  \begin{itemize}
  \item Simple access to big number of existing Java libraries
  \item Clojure-specific libraries:
    \begin{itemize}
    \item Clojure-Contrib -- ``semi-standard'' libraries
    \item Compojure, Ring, Scriptjure -- Web development
    \item ClojureQL -- databases
    \item Incanter -- R-like environment and libraries
    \item other -- see at \url{http://clojars.org}
    \end{itemize}
  \item Repositories:
    \begin{itemize}
    \item build.clojure.org
    \item clojars.org
    \end{itemize}
  \end{itemize}
\note{say more about libraries}
\end{frame}

\begin{frame}[t]
  \frametitle{Commercial projects}
  \begin{itemize}
    \item FlightCaster -- machine learning, etc.
    \item Relevance, Inc. -- consulting
    \item Runa -- marketing services
    \item Sonian Networks -- archiving solutions
    \item BackType -- social media analytics
    \item DRW Trading Group -- trading and finance
    \item DocuHarvest project by Snowtide Informatics Systems, Inc.
    \item ThorTech Solutions -- scalable, mission-critical solutions
    \item TheDeadline project by freiheit.com (\url{http://www.freiheit.com/})
    \item and many more\ldots{}
  \end{itemize}
\note{...}
\end{frame}

\section{Sources of information}

\begin{frame}[t,fragile]
  \frametitle{Sources of information (1)}
  \begin{itemize}
  \item Main sites:
    \begin{itemize}
    \item Site of the language -- \url{http://clojure.org}
    \item Planet Clojure -- \url{http://planet.clojure.in}
    \item the \lstinline|#clojure| IRC channel at freenode.net
    \item The labrepl project (\url{http://github.com/relevance/labrepl}) -- learning
      environment
    \item Try-Clojure (\url{http://www.try-clojure.org/}) -- you can execute Clojure code via Web-browser
    \item \url{http://clojuredocs.org/} -- documentation and examples
    \end{itemize}
  \item German resources:
    \begin{itemize}
    \item Book in German will released in 2010 fall (\url{http://www.clojure-buch.de/})
    \item clojure-de mailing list (\url{http://groups.google.com/group/clojure-de})
    \item videos about Clojure in German (\url{http://www.rheinjug.de/videos/gse.lectures.app/Talk.html#Clojure})
    \end{itemize}
  \end{itemize}
\note{

Mention Concur.next blog series ? }
\end{frame}

\begin{frame}[t]
  \frametitle{Sources of information (2)}
  \begin{itemize}
  \item Books:
    \begin{itemize}
    \item Programming Clojure -- 2009th, Clojure v. 1.0
    \item Practical Clojure. The Definitive Guide -- 2010th, Clojure, v. 1.2
    \item The Joy of Clojure (beta)
    \item Clojure in Action (beta)
    \item Clojure Programming on Wikibooks
    \item Clojure Notes on RubyLearning
    \end{itemize}
  \item Video-lectures \& screencasts 
  \item Clojure user groups around the world
  \item Clojure study courses (on-site in USA \& Europe, or online)
  \end{itemize}
  \note{

    Say, that ``Practical Clojure'' is better than ``Programming Clojure'',
    because it describes version 1.2 and has better description of STM, etc.

    List of video-lectures \& screencasts you can find on my site

}
\end{frame}

\section{Examples}

\begin{frame}[t, fragile]
  \frametitle{Examples. Hadoop}
  \begin{itemize}
  \item Hadoop word count example is only lines of code (using clojure-hadoop), instead
    dozens for Java...
  \item All low-level details are handled by macros
\begin{lstlisting}
(ns clojure-hadoop.examples.wordcount3
  (:import (java.util StringTokenizer)))

(defn my-map [key value]
  (map (fn [token] [token 1])
       (enumeration-seq (StringTokenizer. value))))

(defn my-reduce [key values-fn]
  [[key (reduce + (values-fn))]])
\end{lstlisting}
  \end{itemize}
\note{

you can get very concise code when using macros...

}
\end{frame}

\begin{frame}[t,fragile]
  \frametitle{Examples. Cascalog}
  \begin{itemize}
  \item Domain specific language to query data in Hadoop with joins, aggregates, custom
    operations, subqueries, etc.
  \item Interactive work with data, using Clojure
  \item Available from \url{http://github.com/nathanmarz/cascalog}
  \item Example: query for persons younger than 30:
\begin{lstlisting}
(?<- (stdout) [?person ?age] (age ?person ?age)
               (< ?age 30))
\end{lstlisting}
  \end{itemize}
\note{this package could be used as example of DSL}
\end{frame}

\begin{frame}[t,fragile]
  \frametitle{Examples. Concurrency}
Ants:
  \begin{itemize}
  \item 320 lines of code, with commentaries and documentation
  \item Massive parallel execution without explicit threading
  \item Many independent objects
  \item Graphical interface
  \end{itemize}

Cluster analysis with K-means algorithm:
\begin{itemize}
\item Each cluster represented by agent
\item Automatically uses all available cores
\item \url{http://www.informatik.uni-ulm.de/ni/staff/HKestler/Reisensburg2009/talks/kraus.pdf}
\end{itemize}
\note{

Show ants in action...

Say not so much about second part

}

\end{frame}

\begin{frame}
  \frametitle{Thank you}
  \begin{columns}
    \begin{column}{0.65\textwidth}
\raggedleft       \Huge{\textbf{Questions}} 
    \end{column}
    \begin{column}{0.35\textwidth}
      \pgfimage[width=3cm]{images/question3}
    \end{column}
  \end{columns}
\end{frame}

\end{document}

% LocalWords:  Lisp transients screencasts Incanter Erlang labrepl Multimethods
% LocalWords:  persistent Net Java sets TODO atoms agents refs vars futures STM
% LocalWords:  pmap pcalls Native threads promise Interoperability namespaces
% LocalWords:  destructuring BigDecimals multimethods Metadata metadata IDEs

%%% Local Variables: 
%%% mode: latex
%%% TeX-master: "Clojure-Munster-slides"
%%% End: 

% LocalWords:  Metaprogramming bytecode JVM REPL
