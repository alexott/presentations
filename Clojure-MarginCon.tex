\documentclass[ignorenonframetext]{beamer}

\mode<presentation> {
% \usetheme{JuanLesPins}
  \usetheme{Madrid}
%  \usetheme{Rochester}
  \setbeamercovered{transparent}
%\usefonttheme[onlylarge]{structuresmallcapsserif}
\usefonttheme[onlysmall]{structurebold}
}

%\usepackage[a4paper,hmargin={2.5cm,2cm},vmargin={2cm,2cm},nohead,twoside]{geometry}
\usepackage{mathtext}
\usepackage{ucs}
\usepackage[utf8x]{inputenc}
\usepackage[T2A]{fontenc}
\usepackage[russian,english]{babel}
\hypersetup{unicode,colorlinks=true,bookmarks}

\usepackage{listings}
\input{clojure-def}
\lstset{lang=clojure}

%TODO: подобрать шрифт для листингов
\lstnewenvironment{clojure}[1][]{%
	\lstset{language=clojure,literate=,#1}}
	{}


\title{Язык Clojure}
\subtitle{Небольшое введение}
\author{Alex Ott}
\institute[\lstinline!alexott@gmail.com!]{\url{http://alexott.net}}
\date{MarginCon 2010}

\begin{document}

\selectlanguage{russian}

Это презентация о Clojure для MarginCon\ldots{}

\frame{\titlepage
}

Представиться, рассказать немного о себе....

\begin{frame}[t]
\frametitle{О чем пойдет речь?}
\tableofcontents
\end{frame}

\begin{itemize}
\item В первой части я расскажу о причинах создания нового языка программирования, о его
  возможностях и отличиях от других диалектов Lisp.
\item Вторая часть будет посвящена знакомству с основами языка и его синтаксисом
\item В третьей части будут рассмотрены вопросы взаимодействия с кодом на Java.  Хотя
  стоит отметить, что такое взаимодействие может происходить не только с Java, но и кодом
  для .Net, если используется версия Clojure.Net (сейчас находится в стадии разработки)
\item Четвертая часть будет посвящена рассмотрению подходов к конкуррентному программированию
\item И в последних двух частях мы быстро рассмотрим инфраструктурные вопросы и то, какие
  источники информации могут быть полезными при работе с языком....
\end{itemize}


\section{Что такое Clojure?}

\begin{frame}[t]
  \frametitle{Что такое Clojure?}
  \begin{columns}
    \begin{column}{0.7\textwidth}
  \begin{itemize}
  \item Lisp'ообразный язык, созданный Rich Hickey.  Анонсирован в 2007-м году.
  \item Спроектирован для работы на существующих платформах -- JVM и .Net (в разработке)
  \item Функциональный язык с неизменяемыми, по умолчанию, данными
  \item Упор на поддержку конкурентного выполнения кода
  \item Открытый исходный код и либеральная лицензия
  \end{itemize}
    \end{column}
    \begin{column}{0.3\textwidth}
      \begin{center}
        \pgfimage[width=2.5cm]{images/clojure}
      \end{center}
    \end{column}
  \end{columns}
\end{frame}

JVM (и .Net) были выбраны в качестве целевых платформ поскольку имеют хорошо проработанные
сборщики мусора, JIT-компиляцию и т.п.  Еще одним плюсом является наличие большого объема
библиотек, разработанных для этих платформ, которые доступны для программистов на
Clojure.  Кроме того, выбор платформы часто диктуется заказчиком ПО.

\begin{frame}[t]
  \frametitle{Причины создания нового языка}
  \begin{columns}
    \begin{column}{0.75\textwidth}
  \begin{itemize}
  \item Lisp, но не отягощенный совместимостью с предыдущими версиями/диалектами
  \item Неизменяемость данных и больший упор на ФП
  \item Поддержка конкурентного программирования на уровне языка
  \item Лучшая интеграция с целевыми платформами по сравнению с прямым переносом
    существующих языков
  \end{itemize}
    \end{column}
    \begin{column}{0.25\textwidth}
      \begin{center}
        \pgfimage[width=2.5cm]{images/question4}
      \end{center}
    \end{column}
  \end{columns}
\end{frame}

% ???
\begin{frame}[t]
  \frametitle{Основные возможности}
  \begin{itemize}
  \item Динамически-типизированный язык
  \item Простой синтаксис, как у всех Lisp'ов
  \item Поддержка интерактивной разработки
  \item Спроектирован в терминах абстракций
  \item Метапрограммирование
  \item Мультиметоды и протоколы (версия 1.2)
  \item Компилируется в байт-код целевых платформ
  \item Доступ к большому количеству существующих библиотек
  \end{itemize}
\end{frame}


\begin{frame}[t]
  \frametitle{Отличия от других Лиспов}
  \begin{itemize}
  \item Измененный синтаксис, с меньшим кол-вом скобок
  \item Изменения в наименовании
  \item Больше first-class структур данных -- отображения (map), наборы (set), вектора
  \item Неизменяемые, по умолчанию, данные
  \item Связывание мета-данных с переменными и функциями
  \item Ленивые коллекции
  \item Нет поддержки макросов для процедуры чтения
  \item Регистро-зависимые имена
  \item Отсутствие tail call optimization (ограничение JVM) -- явные циклы \texttt{loop/recur}
  \item Исключения вместо сигналов и рестартов
  \end{itemize}
\end{frame}


\section{Основы языка}

\begin{frame}[t]
  \frametitle{Базовые типы данных}
  \begin{itemize}
  \item Целые числа произвольного размера -- \texttt{14235344564564564564}
  \item Рациональные числа -- \texttt{26/7}
  \item Вещественные числа -- \texttt{1.2345} и BigDecimals -- \texttt{1.2345M}
  \item Строки (String из Java) -- \texttt{"hello world"}
  \item Буквы (Character в Java) -- \texttt{$\backslash{}$a}, \texttt{$\backslash{}$b}, \ldots{}
  \item Регулярные выражения -- \texttt{\#"[abc]*"}
  \item Boolean -- \texttt{true} и \texttt{false}
  \item \texttt{nil} -- также как \texttt{null} в Java, не является пустым списком как в Lisp
  \item Символы (symbol) -- \texttt{test}, \texttt{var1}, \ldots{}
  \item Ключевые символы (keywords) -- \texttt{:test}, \texttt{:hello}, \ldots{}
  \end{itemize}
\end{frame}

\begin{frame}[t,fragile]
  \frametitle{Структуры данных}
  \begin{itemize}
  \item Отдельный синтаксис для разных коллекций:
    \begin{itemize}
    \item Списки -- \texttt{(1 2 3 "abc")}
    \item Вектора -- \texttt{[1 2 3 "abc"]}
    \item Отображения (maps) -- \lstinline!{:key1 1234 :key2 "value"}!
    \item Наборы (sets) -- \lstinline!#{:val1 "text" 1 2 10}!
    \end{itemize}
  \item Последовательность -- абстракция для работы со всеми коллекциями (в том числе и
    классов Java/.Net)
  \item Общая библиотека функций для работы с последовательностями
  \item Ленивые операции над последовательностями
  \item Вектора, отображения, наборы сами являются функциями -- упрощение доступа к данным
  \item Стабильные (persistent) коллекции
  \item ``Переходные'' (transients) коллекции -- оптимизация производительности
  \item \texttt{defstruct} -- оптимизация отображений для объявления сложных структур
  \end{itemize}
\end{frame}

\texttt{defstruct} используется для оптимизации производительности, если вы создаете
большое количество отображений с одинаковой структурой...

\begin{frame}[t]
  \frametitle{Синтаксис языка}
  \begin{itemize}
  \item Простой синтаксис -- программа как структуры данных
  \item Процедура чтения (reader) преобразует программу в структуры данных
  \item Все объекты представляют сами себя за исключением символов и списков
  \item Символы связывают значение с ``переменной''
  \item Списки рассматриваются как формы, которые могут быть:
    \begin{itemize}
    \item Специальной формой -- \texttt{def}, \texttt{let}, \texttt{if}, \texttt{loop}, \ldots{}
    \item Макросом
    \item Функцией, или выражением, которое приводится к функции (maps, keywords, \ldots{})
    \end{itemize}
  \end{itemize}
\end{frame}

Специальные формы -- формы в которых не работает стандартный порядок вычисления аргументов.

\begin{frame}[fragile,fragile]
  \frametitle{Пример кода}
\begin{lstlisting}
(defn fibo
  ([] (concat [1 1] (fibo 1 1)))
  ([a b]
     (let [n (+ a b)]
       (lazy-seq (cons n (fibo b n))))))
(take 100000000 (fibo))

(defn vrange2 [n]
  (loop [i 0 
         v (transient [])]
    (if (< i n)
      (recur (inc i) (conj! v i))
      (persistent! v))))
\end{lstlisting}
\end{frame}

\begin{frame}[t,fragile]
  \frametitle{Функции}
  \begin{itemize}
  \item Функции как first-class objects
  \item Возможность разного кол-ва аргументов в функциях % см. предыдущий слайд, функция fib
  \item Определение функций: \lstinline!defn! -- top-level функции, \lstinline!fn! --
    анонимные функции
  \item Упрощенный синтаксис для анонимных функций -- \lstinline!#(code)!. Например:\\
    \lstinline!(map #(.toUpperCase %) ["test" "hello"])!\\
    \lstinline!(map #(vector %1 %2) [1 2 3] [4 5 6])!
  \item Возможность задания тестов и pre/post-условий в метаданных. Например,
\begin{lstlisting}
(defn constrained-sqr [x]
  {:pre  [(pos? x)] :post [(> % 16), (< % 225)]}
  (* x x))
\end{lstlisting}
  \item Функция компилируется в отдельный класс Java
  \end{itemize}
\end{frame}

\begin{frame}[t,fragile]
  \frametitle{Метапрограммирование и макросы}
  \begin{itemize}
  \item Макрос получает на вход код и возвращает новый код, который будет откомпилирован
  \item Очень большая часть языка написана с помощью макросов
  \item Также как и функции, могут иметь переменной число аргументов
  \item Код можно генерировать с помощью функций для работы со списками, но удобней
    пользоваться quasi-quote -- \lstinline!`! и операторами подстановки -- \lstinline!~! и
    \lstinline!~@! % -- показать разницу на следующих двух примерах
  \item Суффикс \lstinline!#! в именах используется для генерации уникальных имен
  \item \texttt{macroexpand-1} \& \texttt{macroexpand} -- средства отладки
  \end{itemize}
\end{frame}

\lstinline!`! также "нормализует" символы -- делает их fully-qualified

\begin{frame}[fragile,t]
  \frametitle{Примеры макросов}
Стандартный макрос \texttt{when}:
\begin{lstlisting}
(defmacro when
  [test & body]
  (list 'if test (cons 'do body)))  
\end{lstlisting}
при следующем использовании
\begin{lstlisting}
(when (pos? a) 
   (println "positive") (/ b a))
\end{lstlisting}
раскроется в:
\begin{lstlisting}
(if (pos? a) 
  (do 
    (println "positive") 
    (/ b a)))
\end{lstlisting}
\end{frame}

\begin{frame}[fragile,t]
  \frametitle{Примеры макросов}
Стандартный макрос \texttt{and}
\begin{lstlisting}
(defmacro and
  ([] true)
  ([x] x)
  ([x & next]
   `(let [and# ~x]
      (if and# (and ~@next) and#))))  
\end{lstlisting}
раскрывается в разные формы, в зависимости от числа аргументов:
\begin{lstlisting}
user> (macroexpand '(and ))        ==> true
user> (macroexpand '(and (= 1 2))) ==> (= 1 2)
user> (macroexpand '(and (= 1 2) (= 3 3))) ==>
(let* [and__4457__auto__ (= 1 2)] 
  (if and__4457__auto__ 
     (clojure.core/and (= 3 3)) 
     and__4457__auto__))
\end{lstlisting}
\end{frame}

\begin{frame}[fragile]
  \frametitle{Полиморфизм и мультиметоды}
  \begin{itemize}
  \item Расширяемость
  \item Мультиметоды не привязаны к типам/классам
  \item Диспатчеризация на основании нескольких параметров
  \item Решение принимается на основании результата функции диспатчеризации
  \item Отличается от CLOS -- отсутствие \texttt{:before}, \texttt{:after}, \ldots{}
  \item \texttt{defmulti} -- аналог \texttt{defgeneric}.
  \item Определяется как \lstinline!(defmulti func-name dispatch-fn)! + набор определений
    конкретных методов (через \texttt{defmethod})
  \end{itemize}
\end{frame}

\begin{frame}[fragile,t]
  \frametitle{Пример мультиметодов}
Простой пример диспатчеризации по классу:
\begin{lstlisting}
(defmulti m-example class)
(defmethod m-example String [this]
  (println "This is string '" this "'"))
(defmethod m-example java.util.Collection [this]
  (print "This is collection!"))
\end{lstlisting}
и получим при запуске:
\begin{lstlisting}
(m-example "Hello") ==> "This is string 'Hello'"
(m-example [1 2 3]) ==> "This is collection!"
(m-example '(1 2 3)) ==> "This is collection!"
\end{lstlisting}
\end{frame}

\begin{frame}[fragile]
  \frametitle{Более сложный пример диспатчеризации}
\begin{lstlisting}
(defmulti encounter 
     (fn [x y] [(:Species x) (:Species y)]))
(defmethod encounter [:Bunny :Lion] [b l] :run-away)
(defmethod encounter [:Lion :Bunny] [l b] :eat)
(defmethod encounter [:Lion :Lion] [l1 l2] :fight)
(defmethod encounter [:Bunny :Bunny] [b1 b2] :mate)
(def b1 {:Species :Bunny})
(def b2 {:Species :Bunny})
(def l1 {:Species :Lion})
(def l2 {:Species :Lion})

(encounter b1 b2) ==> :mate
(encounter b1 l1) ==> :run-away
(encounter l1 b1) ==> :eat
(encounter l1 l2) ==> :fight
\end{lstlisting}
\end{frame}


\begin{frame}[t]
  \frametitle{Протоколы и типы данных}
  \begin{itemize}
  \item Введены в версии 1.2
  \item Быстрее чем мультиметоды
  \item Позволяют написать Clojure in Clojure
  \item Диспатчеризация только по типу данных
  \item Похожи на type classes в Haskell
  \item Создают соответствующие интерфейсы для Java
  \item \texttt{defrecord} \& \texttt{deftype} объявляют новые типы данных
  \item \texttt{extend-protocol} \& \texttt{extend-type} связывают протокол с типами
    данных (не только объявленными в Clojure!)
  \item \texttt{reify} -- для реализации протоколов и интерфейсов для ``одноразовых'' типов
  \end{itemize}
\end{frame}

\begin{frame}[fragile]
  \frametitle{Пример протоколов}
\begin{lstlisting}
(defprotocol Hello "Test of protocol"
  (hello [this] "hello function"))

(defrecord B [name] Hello
  (hello [this] (str "Hello " (:name this) "!")))
(hello (B. "User")) ==> "Hello User!"

(extend-protocol Hello String
              (hello [this] (str "Hello " this "!")))
(hello "world") ==> "Hello world!"

(extend-protocol Hello java.lang.Object
              (hello [this] (str "Hello '" this 
                     "'! (" (type this) ")")))
(hello 1) ==> "Hello '1'! (class java.lang.Integer)"
\end{lstlisting}
\end{frame}

\begin{frame}[fragile,t]
  \frametitle{Прочее}
  \begin{itemize}
  \item Метаданные -- \lstinline!#^{}! в 1.0 и 1.1 или просто \lstinline!^{}! в 1.2
    \begin{itemize}
    \item Опциональная спецификация типов (type hints) -- \lstinline!#^Type! или \lstinline!^Type!
    \item Спецификация тестов прямо в объявлении функции
    \item Управление областью видимости
    \item Программный доступ к метаданным
    \item Не влияют на равенство (equality) значений
    \end{itemize}
  \item Пространства имен (namespaces)
    \begin{itemize}
    \item First-class objects
    \item Используются для организации кода в библиотеки
    \end{itemize}
  \item Деструктуризация параметров функций и возвращаемых значений
\begin{lstlisting}
(let [[a b & c :as e] [1 2 3 4]] [a b c e]) 
   ==> [1 2 (3 4) [1 2 3 4]]
(let [{:keys [a b c] :as m :or {b 3}} {:a 5 :c 6}]
  [a b c m]) ==> [5 3 6 {:a 5 :c 6}]
\end{lstlisting}
  \end{itemize}
\end{frame}

%TODO: написать во что компилируется кложурный код....

\section{Взаимодействие с Java}

Стоит отметить, что взаимодействие может происходить и с другими целевыми платформами,
например, с .Net

\begin{frame}[t]
  \frametitle{Взаимодействие с Java}
  \begin{itemize}
  \item Двухстороннее взаимодействие с целевой платформой:
    \begin{itemize}
    \item Создание экземпляров классов Java -- \texttt{new} или \texttt{Class.}
    \item Вызов кода, написанного на Java -- \texttt{.}, \texttt{..}, \texttt{doto}
    \item Генерация классов и интерфейсов для вызова из Java -- \texttt{gen-class},
      \texttt{gen-interface}, \texttt{proxy} (анонимный класс)
    \end{itemize}
  \item Возможность выполнения кода Clojure из программ на Java
  \item Отдельные функции для работы с массивами Java
    \begin{itemize}
    \item \texttt{make-array} -- создание массивов
    \item \texttt{aget}/\texttt{aset} -- доступ к элементам массивов
    \item \texttt{amap}/\texttt{areduce} -- итерация по элементам массивов
    \end{itemize}
  \item \texttt{memfn} позволяет использовать функции-члены классов в \texttt{map},
    \texttt{filter}, \ldots{}
  \item Спец. форма \texttt{set!} для установки значений в классе
  \item Генерация и перехват исключений -- \texttt{throw} \& \texttt{catch}
  \end{itemize}
\end{frame}

Основными формами являются \texttt{.} и \texttt{new}, остальное добавлено для упрощения
жизни... 

\begin{frame}[fragile]
  \frametitle{Примеры}
  \begin{itemize}
  \item Создание объектов:
\begin{lstlisting}
(new java.util.Date) <==> (java.util.Date.)
\end{lstlisting}
  \item Доступ к полям/функциям членам классов:
\begin{lstlisting}
(.substring "Hello World" 0 5) ==> "Hello"
(. "Hello World" substring 0 5) ==> "Hello"
Math/PI  ==> 3.141592653589793
(Integer/parseInt "42") ==> 42
\end{lstlisting}
  \item \texttt{..} для связанных вызовов:
\begin{lstlisting}
(.. System getProperties (get "os.name")) ==> "Mac OS X"
System.getProperties().get("os.name")
\end{lstlisting} % Кто сказал что в лиспе больше скобок чем в яве?
  \item \texttt{doto} -- вызов нескольких методов для одного объекта:
\begin{lstlisting}
(doto (java.util.HashMap.)
 (.put "a" 1) (.put "b" 2))
\end{lstlisting}
  \end{itemize}
\end{frame}

% \begin{frame}
%   \frametitle{ПрГенерация классов}
%TODO: gen-class, gen-interface, proxy - когда что используется
%TODO: add example of generated class? or just use proxy example?
% \end{frame}

\section{Конкурентное программирование}

\begin{frame}[t,fragile]
  \frametitle{Конкурентное программирование}
  \begin{itemize}
  \item Средства, обеспечивающие изменяемость данных:
    \begin{itemize}
    \item Ссылки (refs) -- синхронное, координированное изменение
    \item Агенты (agents) -- асинхронное, независимое
    \item Атомы (atoms) -- синхронное, независимое
    \item Переменные (vars) -- изменение, локальное для нитей
    \item \lstinline|@| (\texttt{deref}) -- используется для доступа к данным
    \end{itemize}
  \item Параллельное выполнение кода:
    \begin{itemize}
    \item futures
    \item \texttt{pmap} \& \texttt{pcalls}
    \item Native threads
    \end{itemize}
  \item Средства синхронизации -- promise
  \end{itemize}
\end{frame}


\begin{frame}[fragile,t]
  \frametitle{Ссылки \& STM}
  \begin{itemize}
  \item Координированное изменение данных из нескольких потоков
  \item Software Transaction Memory обеспечивает атомарность, целостность, изоляцию
  \item Изменения происходят только в рамках транзакции
  \item Возможность проверки данных с помощью функции-валидатора
  \item Возможность добавления функций-наблюдателей
  \end{itemize}
\begin{lstlisting}
(def counters (ref {}))
(defn get-counter [key]
  (@counters key 0))
(defn increment-counter [key]
  (dosync
   (alter counters assoc key 
          (inc (@counters key 0)))))
\end{lstlisting}
\end{frame}

\begin{frame}[fragile,t]
  \frametitle{Агенты}
  \begin{itemize}
  \item Асинхронное обновление данных -- ``fire and forget''
  \item Пулы потоков для выполнения функций обновления данных: \texttt{send} -- для
    ``быстрого'' обновления данных, \texttt{send-off} -- для ``тяжелых'' функций
    (отдельный пул потоков выполнения)
  \item \texttt{send} \& \texttt{send-off} получают функцию, которая будет применена к
    текущему состоянию агента
  \item Функции-валидаторы и функции-наблюдатели
  \item Возможность ожидания окончания всех заданий агента
  \end{itemize}
\begin{lstlisting}
(def acounters (agent {}))
(defn increment-acounter [key]
  (send acounters assoc key 
        (inc (@counters key 0))))
(defn get-acounter [key]
  (@acounters key 0))
\end{lstlisting}
\end{frame}

\begin{frame}[t,fragile]
  \frametitle{Vars \& Атомы}
  \begin{itemize}
  \item Атомы
    \begin{itemize}
    \item Синхронное изменение данных без координации
    \item Изменение данных производится функцией, которая применяется к текущему значению
    \item Функция может быть вызвана несколько раз, если кто-то уже изменил значение
    \item Функция не должна иметь побочных эффектов!
    \end{itemize}
  \item Vars
    \begin{itemize}
    \item Обеспечивают изменение данных в рамках текущего потока
    \item \texttt{binding} связывает новые значения с символами
    \item Изменения затрагивают все вызываемые из текущего потока функции
    \item Будьте осторожны с ленивыми последовательностями!
    \end{itemize}
\begin{lstlisting}
(def *var* 5)
(defn foo [] *var*)
(foo) ==> 5
(binding [*var* 10] (foo)) ==> 10
\end{lstlisting}
  \end{itemize}
\end{frame}

\begin{frame}[t,fragile]
  \frametitle{Параллельное выполнение кода}
  \begin{itemize}
  \item \texttt{future}
    \begin{itemize}
    \item выполняет заданный код в отдельном потоке
    \item \lstinline|@| -- для доступа к результатам выполнения кода
    \item \lstinline|@| блокирует текущий поток, если нет результатов
    \end{itemize}
  \item \texttt{promise}
    \begin{itemize}
    \item используется для синхронизации между потоками данных
    \item \texttt{deliver} устанавливает значение
    \item \lstinline|@| читает установленное значение или блокирует выполнение, если оно
      не установлено
    \end{itemize}
  \item \texttt{pmap} \& \texttt{pcalls} -- выгодно использовать только для ``тяжелых''
    функций обработки данных.
  \end{itemize}
\end{frame}

\section{Clojure в реальной жизни}

\begin{frame}[t]
  \frametitle{Clojure в реальной жизни}
  \begin{itemize}
  \item Инфраструктура и инструментальная поддержка:
    \begin{itemize}
    \item Среды разработки
    \item Средства сборки
    \item Библиотеки
    \item Репозитории кода
    \end{itemize}
  \item Clojure/core -- коммерческая поддержка, консультации и т.п.
  \item Использование в коммерческих проектах
  \item Источники информации
  \end{itemize}
\end{frame}

\begin{frame}[t]
  \frametitle{Инфраструктура: IDE и средства сборки}
  \begin{itemize}
  \item Поддержка в большинстве IDE:
  \begin{itemize}
  \item Emacs + SLIME (самый популярный)
  \item VimClojure
  \item NetBeans
  \item Eclipse
  \item IntelliJ IDEA
  \end{itemize}
  \item Утилиты сборки кода:
    \begin{itemize}
    \item Поддержка Clojure в Maven и Ant
    \item Leiningen -- написан на Clojure, расширяемый и очень простой в использовании
    \end{itemize}
  \end{itemize}
\end{frame}

\begin{frame}[t]
  \frametitle{Инфраструктура: библиотеки и репозитории}
  \begin{itemize}
  \item Доступ к большому набору существующих библиотек
  \item Clojure-specific libraries:
    \begin{itemize}
    \item Clojure-Contrib
    \item Compojure
    \item ClojureQL
    \item Incanter
    \item и другие -- см. \url{http://clojars.org}
    \end{itemize}
  \item Репозитории:
    \begin{itemize}
    \item build.clojure.org
    \item clojars.org
    \end{itemize}
  \end{itemize}
\end{frame}

\begin{frame}[t]
  \frametitle{Использование в коммерческих проектах}
  \begin{itemize}
  \item Зарубежные проекты:
    \begin{itemize}
    \item FlightCaster
    \item ThinkRelevance
    \item Runa
    \item Sonian Networks
    \item BackType
    \item DRW Trading Group
    \item Snowtide Informatics Systems, Inc. - проект DocuHarvest
    \item ThorTech Solutions
    \end{itemize}
  \item Несколько российских стартапов используют Clojure:
    \begin{itemize}
    \item ООО "Моделирование и Технологии"
    \item Security Technology Research (\url{http://setere.com})
    \end{itemize}
  \end{itemize}
\end{frame}

\section{Источники информации}

\begin{frame}[t,fragile]
  \frametitle{Источники информации}
  \begin{itemize}
  \item Основные:
    \begin{itemize}
    \item Сайт проекта -- \url{http://clojure.org}
    \item Planet Clojure -- \url{http://planet.clojure.in}
    \item Канал \lstinline|#clojure| на freenode.net
    \item Проект labrepl (\url{http://github.com/relevance/labrepl}) -- учебная среда для
      изучения языка
    \item Try-Clojure (\url{http://www.try-clojure.org/}) -- работа с кодом используя
      только Web-браузер
    \end{itemize}
  \item Русскоязычные ресурсы:
    \begin{itemize}
    \item Русская планета ФП -- \url{http://fprog.ru/planet/}
    \item clojure at conference.jabber.ru
    \item Clojure форум на \url{http://lisper.ru}
    \item Мой сайт -- \url{http://alexott.net/ru/clojure/}
    \end{itemize}
  \end{itemize}
\end{frame}

\begin{frame}[t]
  \frametitle{Источники информации}
  \begin{itemize}
  \item Книги:
    \begin{itemize}
    \item Programming Clojure -- 2009-й год, версия 1.0
    \item Practical Clojure. The Definitive Guide -- 2010, версия 1.2
    \item The Joy of Clojure (beta)
    \item Clojure in Action (beta)
    \item Clojure Programming on Wikibooks
    \item Clojure Notes on RubyLearning
    \end{itemize}
  \item Видео-лекции и скринкасты
  \item Группы пользователей по всему миру
  \item Учебные курсы (пока только в США и Европе)
  \end{itemize}
\end{frame}

\begin{frame}[t]
  \frametitle{Спасибо за внимание}
  \begin{columns}
    \begin{column}{0.65\textwidth}
\raggedleft       \Huge{\textbf{Вопросы}} 
    \end{column}
    \begin{column}{0.35\textwidth}
      \pgfimage[width=3cm]{images/question3}
    \end{column}
  \end{columns}
\end{frame}

\end{document}

%%% Local Variables: 
%%% mode: latex
%%% TeX-master: t
%%% End: 

% LocalWords:  Мультиметоды Метапрограммирование Lisp мультиметоды transients
% LocalWords:  persistent Net Java sets TODO atoms agents refs vars futures STM
% LocalWords:  pmap pcalls Native threads promise Interoperability namespaces
% LocalWords:  destructuring BigDecimals диспатчеризации
